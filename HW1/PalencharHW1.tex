\title{Homework One}
\author{
        John E. Palenchar \\
                Department of Mathamatics\\
        Student---University of Florida\\
}
\date{\today}

\documentclass[12pt]{article}
% Required package
\usepackage{amsmath}
\usepackage{amssymb}
\usepackage{latexsym}
\begin{document}

%This is the title maker
\maketitle

%start problem 1
\section*{Problem 1}\label{Problem 1}
\begin{center}
        $$
                \text{Find the sets } A,B,C \text{ } : A - (B \cup C) \neq (A-B) \cup C
        $$\\
\end{center}
\underline{ANSWER} \\
\begin{center}
        Suppose that $A \subseteq C \subseteq B$ \\
        To start let us define our set operations as\\
        $$
                x \in A - B \iff x \in A \land x \notin B
        $$
        $$
                x \in A \cup B \iff x \in A \lor x \in B
        $$
        $$
                x \in A \cap B \ \iff \ x \in A \land \ x \in B
        $$
\end{center}
One can say that the left side of the equation is\\

\begin{math}
        x \in A - (B\cup C) \Leftrightarrow \\
        x \in A \land x \notin (B \cup C) \Leftrightarrow \\
        x \in A \land (x \notin B \lor x \notin C) \Leftrightarrow\\
        (x \in A \land x \notin B) \lor (x \in A \land x \notin C) \\
\end{math}\\
One can then take the left side of the equation and write it as \\
\begin{math}
        x \in (A-B)\cup C \Leftrightarrow \\
        x \in (A-B) \lor x \in C \Leftrightarrow \\
        (x \in A \land x \notin B) \lor x \in C \Leftrightarrow \\
        (x \in C \lor x \in A) \lor (x \in C \land x \notin B)
\end{math}\\
At this point there is a contradiction because on the left hand one get that $x \in C \lor x \in A$. Where as on the right side one get that $x \in A \land x \notin C$.\\
\begin{flushright}
        $\Box$
\end{flushright}

From this we can find that there are some sets that exemplify that $A-(B\cup C) \neq (A-B)\cup C$\\
$A = \{1,2,3,4,5\}$\\
$B = \{-1,0,1,2\}$\\
$C = \{3,4,5,6,7\}$
When working these out we get that the left side is equal to the empty set. While the Right side equals the set $\{3,4,5,6,7\}$

\section*{Problem 2}\label{Problem 2}
\begin{center}
        Consider the set $D = {7x+3y : x \in \mathbb{Z}, y \in \mathbb{Z}}$. \\
        (a) Show that $1 \in D$ holds.\\
        (b) Use (a) to show that $D = \mathbb{Z}$ \\
\end{center}
\underline{ANSWER (a)}\\
\\Let one take the equation $7x+3y$ and write it as $7x+3Y = 1$.
One knows that $x \text{ and } y$ can be any number in $\mathbb{Z}$.
One can now make $x=10$ and $y=-23$ as both $10$ and $-23$ are integers this shows that $1 \in D$\\
\\\underline{ANSWER (B)}\\

One can define a function  $f(w) = 7x+3(-2w)$ where $w \in \mathbb{Z}$.
The range of this function is $\mathbb{Z}$. From this it can be said that $D = \mathbb{Z}$
\begin{flushright}
        $\Box$
\end{flushright}

\end{document}